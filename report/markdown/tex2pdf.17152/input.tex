\documentclass[12pt,,]{report}
\usepackage{lmodern}
\usepackage{amssymb,amsmath}
\usepackage{ifxetex,ifluatex}
\usepackage{fixltx2e} % provides \textsubscript
\ifnum 0\ifxetex 1\fi\ifluatex 1\fi=0 % if pdftex
  \usepackage[T1]{fontenc}
  \usepackage[utf8]{inputenc}
\else % if luatex or xelatex
  \ifxetex
    \usepackage{mathspec}
  \else
    \usepackage{fontspec}
  \fi
  \defaultfontfeatures{Ligatures=TeX,Scale=MatchLowercase}

    \usepackage{xeCJK}
    % 中文自動換行
    \XeTeXlinebreaklocale "zh"
    % 文字的彈性間距
    \XeTeXlinebreakskip = 0pt plus 1pt
    \newfontlanguage{Chinese}{CHN}
    % 章次20級,節次16級,小節次以下14級,本文12級字
    \def\LARGE{\fontsize{20}{30}\selectfont}%章次
    \def\Large{\fontsize{16}{24}\selectfont}%節次
    \def\large{\fontsize{14}{21}\selectfont}%小節次
    \usepackage{indentfirst}
    \usepackage{CJKnumb}
    \renewcommand{\figurename}{圖}
    \renewcommand{\thefigure}{{\arabic{chapter}}.\arabic{figure}}
    \renewcommand{\tablename}{表}
    \renewcommand{\thetable}{{\arabic{chapter}}.\arabic{table}}
    %重製章節
    \renewcommand{\chaptername}{}
    \renewcommand{\thechapter}{第\CJKnumber{\arabic{chapter}}章}
    \renewcommand{\thesection}{{\arabic{chapter}}.\arabic{section}}
    \renewcommand{\thesubsection}{{\arabic{chapter}}.{\arabic{section}}.\arabic{subsection}}
    %設定行距與中英文字型
    \linespread{1}\selectfont
    \setCJKmainfont{SimSun}
    \setmainfont{Times New Roman}
    \setromanfont{Times New Roman}
    \setmonofont{Times New Roman}
    %重製章節標籤
    \usepackage{titlesec}
    \titleformat{\chapter}[block]{\LARGE\centering}{\thechapter}{0.5em}{}
    \titleformat{\section}[block]{\Large}{\thesection}{0.5em}{}
    \titleformat{\subsection}[block]{\large}{\thesubsection}{0.5em}{}
    % 重製目錄
    \usepackage{titletoc}
    \titlespacing{\chapter}{0pt}{*0}{*2}
    \titlespacing{\section}{0pt}{*1}{*1}
    \titlespacing{\subsection}{0pt}{*1}{*1}
    \titlespacing{\subsubsection}{0pt}{*1}{*1}
    \titlecontents{chapter}[0em]{}{\contentspush{\thecontentslabel}\hspace*{1em}}{}{\titlerule*[0.7pc]{.}\contentspage}
\fi
% use upquote if available, for straight quotes in verbatim environments
\IfFileExists{upquote.sty}{\usepackage{upquote}}{}
% use microtype if available
\IfFileExists{microtype.sty}{
\usepackage{microtype}
\UseMicrotypeSet[protrusion]{basicmath} % disable protrusion for tt fonts
}{}
\usepackage[margin=1in]{geometry}
\usepackage[unicode=true]{hyperref}
\hypersetup{
            pdfauthor={設計一乙 40623237; 設計一乙 40623238; 設計一乙 40623239; 設計一乙 40623246; 設計一乙 40623247; 設計一乙 40623248},
            pdfborder={0 0 0},
            breaklinks=true}
\urlstyle{same}  % don't use monospace font for urls
\ifnum 0\ifxetex 1\fi\ifluatex 1\fi=0 % if pdftex
  \usepackage[shorthands=off,main=]{babel}
\else
  \usepackage{polyglossia}
  \setmainlanguage[]{}
\fi
\usepackage{graphicx,grffile}
\makeatletter
\def\maxwidth{\ifdim\Gin@nat@width>\linewidth\linewidth\else\Gin@nat@width\fi}
\def\maxheight{\ifdim\Gin@nat@height>\textheight\textheight\else\Gin@nat@height\fi}
\makeatother
% Scale images if necessary, so that they will not overflow the page
% margins by default, and it is still possible to overwrite the defaults
% using explicit options in \includegraphics[width, height, ...]{}
\setkeys{Gin}{width=\maxwidth,height=\maxheight,keepaspectratio}
\IfFileExists{parskip.sty}{%
\usepackage{parskip}
}{% else
\setlength{\parindent}{0pt}
\setlength{\parskip}{6pt plus 2pt minus 1pt}
}
\setlength{\emergencystretch}{3em}  % prevent overfull lines
\providecommand{\tightlist}{%
  \setlength{\itemsep}{0pt}\setlength{\parskip}{0pt}}
\setcounter{secnumdepth}{5}
% Redefines (sub)paragraphs to behave more like sections
\ifx\paragraph\undefined\else
\let\oldparagraph\paragraph
\renewcommand{\paragraph}[1]{\oldparagraph{#1}\mbox{}}
\fi
\ifx\subparagraph\undefined\else
\let\oldsubparagraph\subparagraph
\renewcommand{\subparagraph}[1]{\oldsubparagraph{#1}\mbox{}}
\fi

% set default figure placement to htbp
\makeatletter
\def\fps@figure{htbp}
\makeatother


\begin{document}
%Cover Start
\begin{titlepage}
\vspace{1cm}
\begin{center}
\fontsize{36}{54}\selectfont{
    國立虎尾科技大學\par
}
\fontsize{28}{42}\selectfont{機械設計工程系\par}
\fontsize{24}{36}\selectfont{計算機程式 bg7 期末報告\par}
\vspace{1.5cm}
\fontsize{20}{30}\selectfont{
    PyQt5 事件導向計算器\par
    PyQt5 Event-Driven Calculator Project\par
}
\vspace{\fill}
\fontsize{18}{27}\selectfont{
    學生:\par
    設計一乙 40623237 \par 設計一乙 40623238 \par 設計一乙 40623239 \par 設計一乙 40623246 \par 設計一乙 40623247 \par 設計一乙 40623248 \par
    指導教授:嚴家銘\par
}
\vspace{1.5cm}
\fontsize{16}{24}\selectfont{2017.12.18\par}
\end{center}
\vspace{1cm}
\end{titlepage}

\newcommand\frontmatter{
    \cleardoublepage
    \pagenumbering{roman}
}

\newcommand\mainmatter{
    \cleardoublepage
    \pagenumbering{arabic}
}

\newcommand\backmatter{
    \if@openright
        \cleardoublepage
    \else
        \clearpage
    \fi
}

%Document start

% Set page number to arabic i ii...
\frontmatter
\renewcommand{\abstractname}{\LARGE \center 摘要}
\chapter*{摘要}
\addcontentsline{toc}{chapter}{摘要}
\fontsize{14}{21}\selectfont{這裡是摘要內容。A pipe character, followed by an indented block of text
is treated as a literal block, in which newlines are preserved
throughout the block, including the final newline.

\begin{itemize}
\tightlist
\item
  以 YAML 的方式插入。
\item
  The `+' indicator says to keep newlines at the end of text blocks.
\item
  使用 Markdown 語法。
\item
  前面使用加號
\end{itemize}

本研究的重點在於 \ldots{}}


\begingroup
    \renewcommand{\contentsname}{\center 目錄 \addcontentsline{toc}{chapter}{目錄}}
    \renewcommand{\numberline}[1]{~#1\hspace*{1em}}
        \setcounter{tocdepth}{2}
    \tableofcontents
    \newcommand{\lotlabel}{表}
    \renewcommand{\listtablename}{\center 表目錄 \addcontentsline{toc}{chapter}{表目錄}}
    \renewcommand{\numberline}[1]{\lotlabel~#1\hspace*{1em}}
    \listoftables
    \newcommand{\loflabel}{圖}
    \renewcommand{\listfigurename}{\center 圖目錄 \addcontentsline{toc}{chapter}{圖目錄}}
    \renewcommand{\numberline}[1]{\loflabel~#1\hspace*{1em}}
    \listoffigures
\endgroup

% Start normal page number, 1 2 3
\mainmatter
\hypertarget{ux524dux8a00}{%
\chapter{前言}\label{ux524dux8a00}}

第七組計算器程式期末報告前言

前言內容: 這是機械設計一乙計算機程式第七組報告。

~

\hypertarget{ux53efux651cux7a0bux5f0fux7cfbux7d71ux4ecbux7d39}{%
\chapter{可攜程式系統介紹}\label{ux53efux651cux7a0bux5f0fux7cfbux7d71ux4ecbux7d39}}

可攜程式系統介紹

\hypertarget{ux555fux52d5ux8207ux95dcux9589}{%
\section{啟動與關閉}\label{ux555fux52d5ux8207ux95dcux9589}}

Windows 的內容

有一張圖片:

\begin{figure}
\centering
\includegraphics{./tex2pdf.17152/e0ef408d9559203849a0aa26f79f9b032b709c7a.png}
\caption{Kmol\label{fig:駱駝}}
\end{figure}

稱為圖 \ref{fig:駱駝}。

各 md 檔案可以在 images 目錄下自訂與 md 檔案名稱相同的子目錄存放影像檔案

\hypertarget{ux555fux52d5ux8207ux95dcux95892}{%
\section{啟動與關閉2}\label{ux555fux52d5ux8207ux95dcux95892}}

\hypertarget{python-ux7a0bux5f0fux8a9eux6cd5}{%
\chapter{Python 程式語法}\label{python-ux7a0bux5f0fux8a9eux6cd5}}

Python 程式語法

\hypertarget{ux8b8aux6578ux547dux540d}{%
\section{變數命名}\label{ux8b8aux6578ux547dux540d}}

IPv4 的內容

有一張圖片:

\begin{figure}
\centering
\includegraphics{./tex2pdf.17152/e0ef408d9559203849a0aa26f79f9b032b709c7a.png}
\caption{Kmol\label{fig:駱駝}}
\end{figure}

稱為圖 \ref{fig:駱駝}。

各 md 檔案可以在 images 目錄下自訂與 md 檔案名稱相同的子目錄存放影像檔案

\hypertarget{print-ux51fdux5f0f}{%
\section{print 函式}\label{print-ux51fdux5f0f}}

\hypertarget{ux91cdux8907ux8ff4ux5708}{%
\section{重複迴圈}\label{ux91cdux8907ux8ff4ux5708}}

\hypertarget{ux5224ux65b7ux5f0f}{%
\section{判斷式}\label{ux5224ux65b7ux5f0f}}

\hypertarget{ux6578ux5217}{%
\section{數列}\label{ux6578ux5217}}

\hypertarget{pyqt5-ux7c21ux4ecb}{%
\chapter{PyQt5 簡介}\label{pyqt5-ux7c21ux4ecb}}

說明 PyQt5 基本架構與程式開發流程

\hypertarget{pyqt5-ux67b6ux69cb}{%
\section{PyQt5 架構}\label{pyqt5-ux67b6ux69cb}}

大部分所見的圖形化介面程式語言都是由有物件導向的程式語言開發的,例如
C++、Java、C\#、Python 等。

直接使用 C++
語言「寫出」圖形介面是一件滿費心的差事,因此有滿多圖形介面的函式庫
(library) 可以使用,例如 Qt、Tk、wxWidgets、GTK+ 等。使用簡單的函式
(function)
就可創造視窗介面,並且有很多函式庫盡力克服「跨平台」的障礙。由於可以包含的部件極多,稱得上圖形介面「框架
(framework)」一詞。

其中 Qt 是由 Qt Project 開發。Qt 支援平台種類眾多,除了常見的
Windows、Linux、Mac 以外,還有非 X Window System
的作業系統。授權方面也十分自由,採用 GNU 較寬鬆通用公共許可證 (GNU
Lesser General Public License, LGPL)、GNU 通用公共許可證 (GNU General
Public License, GPL)、商業授權三種模式,可以讓開發者應需求選擇。

Qt
程式庫中甚至支援開發圖形介面的「周邊」功能,如網路通訊、OpenGL、OpenVG、SQL
與 XML 直譯器、圖片格式轉檔、Linux 的輸入法開發、瀏覽器引擎(使用 Google
Chromium)、各式圖表等。

現在 Qt 的版本來到了第 5 版,而且每 6
個月仍再持續提出更新計畫,在自由軟體產業具有很高的影響力。

由於 Qt 的功能極為強大,英國的 Riverbank Computing 公司率先為其撰寫
Python 語言的套件,甚至開發了 SIP 這套工具將 C 與 C++ 程式庫包裝為
Python 套件。

PyQt 幾乎支援 Qt 大部分的功能,並且將較專門的功能另外分成 PyQt Chart(2D
圖表)、PyQt Data Visualization(3D 圖表)、PyQt
Purchasing(應用程式購買功能)。

另外 QScintilla 是一個將 Scintilla 連結至 PyQt 的套件(在 C++ 可以直接用
Qt 和 Scintilla 即可),用途是辨識文字中的程式語言,以亮顯 (highlight)
的方式呈現,可以用作程式語言的辨識功能。

PyQt 的版本與 Qt 相同(除了小版號),採用 GPL
和商業授權。需要注意的是,若作為軟體釋出,沒有商業授權是需要公開原始碼的。

\hypertarget{ux5fc3ux5f97}{%
\chapter{心得}\label{ux5fc3ux5f97}}

期末報告心得

\hypertarget{fossil-scm}{%
\section{Fossil SCM}\label{fossil-scm}}

剛開始接觸都不太熟,甚至連fossil也不知道是甚麼。但經過幾個禮拜老師的教學後,不但都能了解,也能實際應用。
Fossil SCM相關指令: fossil init 倉儲名稱.fossil fossil ui fossil open
fossil clone fossil add . fossil commit -m --no-warnings fossil update

\hypertarget{ux7db2ux8a8cux5fc3ux5f97}{%
\section{網誌心得}\label{ux7db2ux8a8cux5fc3ux5f97}}

第一次上傳到自己的網誌總是困難重重,近端遠端常常上傳不了。後來與同學老師互相討論才擁有現在管理得不錯的網誌。

\hypertarget{github-ux5354ux540cux5009ux5132}{%
\section{Github 協同倉儲}\label{github-ux5354ux540cux5009ux5132}}

與組員的協同總是特別好,很有分工合作的感覺,當完成屬於大家的計算機後成就感十足。

\hypertarget{ux5b78ux54e1ux5fc3ux5f97}{%
\section{學員心得}\label{ux5b78ux54e1ux5fc3ux5f97}}

剛開始協同時常常出問題,後來藉由組長的教學還有老師的影片,才慢慢了解其中操作的過程。
by40623246
在編寫算計機時,對於程式語言還好不了解,後來一直反覆觀看影片以及詢問組長,才慢慢了解。
by40623247

\hypertarget{ux7d50ux8ad6}{%
\chapter{結論}\label{ux7d50ux8ad6}}

期末報告結論

\hypertarget{ux7d50ux8ad6ux8207ux5efaux8b70}{%
\section{結論與建議}\label{ux7d50ux8ad6ux8207ux5efaux8b70}}

結論與建議內容

\hypertarget{ux53c3ux8003ux6587ux737b}{%
\chapter{參考文獻}\label{ux53c3ux8003ux6587ux737b}}


\end{document}
